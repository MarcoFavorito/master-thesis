\chapter{FLLOAT}\label{ch:flloat}
In this chapter we describe \href{https://github.com/MarcoFavorito/flloat.git}{FLLOAT} (From \LLf tO AutomaTa), a software project written in Python. It is a porting of the \href{https://github.com/RiccardoDeMasellis/FLLOAT}{homonym software project} written in Java. It is the implementation of many of the topics described in Chapter \ref{ch:logic}. 

%We'll see the main features, the package structure, and some code examples.

\section{Introduction}
\paragraph{Main features:} FLLOAT is a Python library that provides support for:
\begin{itemize}
	
	\item Syntax, semantics and parsing of the following logic formalisms:
	\begin{itemize}
		\item Propositional Logic;
		\item Linear Temporal Logic on Finite Traces \LTLf
		\item Linear Dynamic Logic on Finite Traces \LDLf;
	\end{itemize}
	\item Conversion from \LLf formula to \NFA, \DFA and \DFA On-The-Fly
	
\end{itemize}

\paragraph{Dependencies:} FLLOAT requires Python>=3.5 and depends on the following packages:
\begin{itemize}
	\item \href{http://www.dabeaz.com/ply/ply.html}{PLY}, a pure-Python implementation of the popular compiler construction tools \href{http://dinosaur.compilertools.net/}{Lex and Yacc}. It has been used for the parsing of \PL and \LLf formulas;
	\item \href{https://github.com/MarcoFavorito/pythomata}{Pythomata}, a Python package which provides support for \NFA, \DFA, determinization and minimization algorithms and reasoning on \DFAs. It has been used for deal with $\automaton_\varphi$, the equivalent automaton of a \LLf formula.
\end{itemize}

\paragraph{Installation:} You can find the package on \href{https://pypi.org/project/flloat/}{PyPI}, hence you can install it with:
\begin{lstlisting}[language=bash]
pip install flloat
\end{lstlisting}
Please go \href{https://github.com/MarcoFavorito/flloat#install}{here} for further details.


\section{Package structure}
The package is structured as follows:
\begin{itemize}
	\item \texttt{flloat.py}: the main module, it contains the implementation of the translation from \LLf formulas to automata. The functions implemented here are called from methods of \LLf formulas.
	\item \texttt{base/}: contains the abstract definitions used in other modules. The main modules are:
	\begin{itemize}
		\item \texttt{Symbol.py} and \texttt{Symbols.py}, where have been defined the class \texttt{Symbol} to represent the atomic propositional symbols and the operator symbols;
		\item \texttt{Alphabet.py}, which is an abstraction for manage a set of \texttt{Symbol};
		\item \texttt{Interpretation.py}, an abstract class denoting the semantics used for truth evaluation. E.g. for \PL the corresponding interpretation is \texttt{PLInterpretation} (a set of \texttt{Symbol}), whereas for \LLf we have \texttt{FiniteTrace}, which is a list of \texttt{PLInterpretation}.
		\item \texttt{Formula.py}, the module containing the base class \texttt{Formula}. Every formula class extends \texttt{Formula}. In this module are defined also \texttt{AtomicFormula}, \texttt{Operator}, \texttt{BinaryOperator} etc., and how to build a syntactic tree. 
		\item \texttt{truths.py} and \texttt{nnf.py} that provide abstract implementations for truth evaluations of formulas and negation normal form operations.
		\item other abstraction definitions that are implemented for each extending subclass.
	\end{itemize}
	\item \texttt{syntax/}: modules for each formalism (i.e. \texttt{pl.py}, \texttt{ltlf.py} and \texttt{ldlf.py}). In those modules are declared all the classes for representing formulas, implementing their truth evaluation procedure  taking into account their correlation (e.g. \texttt{And} is the negative dual of \texttt{Or}, you can define \texttt{Implies} in terms of \texttt{Not} and \texttt{Or} etc.);
	\item \texttt{semantics/}: modules providing implementations for the semantics. E.g. you can find \texttt{PLInterpretation} and \texttt{FiniteTrace} cited before;
	\item \texttt{parser/}: modules where are defined the parsers of formulas in \PL and \LLf. They depends on PLY.
\end{itemize}

\section{Code examples}


Parse a \LDLf formula:

\begin{lstlisting}[language=Python, style=Python]
from flloat.parser.ldlf import LDLfParser

parser = LDLfParser()
formula = "<true*; A & B>tt"
# returns a LDLfFormula
parsed_formula = parser(formula)        

# prints "<((true)* ; (A & B))>(tt)"
print(parsed_formula)                   
# prints {A, B}
print(parsed_formula.find_labels())
     
\end{lstlisting}
Evaluate it over finite traces:

\begin{lstlisting}[language=Python, style=Python]
from flloat.semantics.ldlf import FiniteTrace

t1 = FiniteTrace.fromStringSets([
{},
{"A"},
{"A"},
{"A", "B"},
{}
])
parsed_formula.truth(t1, 0)  # True
\end{lstlisting}
Transform it into an automaton (pythomata.DFA object):

\begin{lstlisting}[language=Python, style=Python]
dfa = parsed_formula.to_automaton(determinize=True)

# print the automaton
dfa.to_dot("./automaton.DFA")
\end{lstlisting}

Notice: \texttt{to\_dot} requires Graphviz. For info about how to use a pythomata.DFA please look at the docs.

The same for a \LTLf formula:

\begin{lstlisting}[language=Python, style=Python]
from flloat.parser.ltlf import LTLfParser
from flloat.base.Symbol import Symbol
from flloat.semantics.ldlf import FiniteTrace

# parse the formula
parser = LTLfParser()
formula = "F(A & !B)"
parsed_formula = parser(formula)

# evaluate over finite traces
t1 = FiniteTrace.fromStringSets([
{},
{"A"},
{"A"},
{"A", "B"}
])
assert parsed_formula.truth(t1, 0)

# from LTLf formula to DFA
dfa = parsed_formula.to_automaton(determinize=True)
assert dfa.word_acceptance(t1.trace)
\end{lstlisting}
 
\section{License}
The software is released under \href{https://github.com/MarcoFavorito/flloat/blob/master/LICENSE}{MIT license}.